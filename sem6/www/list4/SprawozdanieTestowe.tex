\documentclass{article}
\usepackage[utf8]{inputenc}
\usepackage{polski}
\usepackage{listings}
\usepackage{xcolor}

\title{Testy API Biblioteki - Lista 4 WWW}
\author{Michał Kallas}
\date{}

\begin{document}

\maketitle

\section{Wstęp}
Testy zostały zaimplementowane przy użyciu frameworka testowego \textbf{Jest}, który jest popularnym narzędziem do testowania aplikacji napisanych w JavaScript.
\textbf{Jest} zapewnia prostą konfigurację, wsparcie dla asynchronicznych testów oraz wbudowane mechanizmy do mockowania i asercji.
Do zarządzania danymi testowymi wykorzystano \textbf{MongoDB Memory Server}, czyli lekką wersję bazy MongoDB działającą w pamięci RAM, co pozwala na szybkie i izolowane testy bez konieczności uruchamiania zewnętrznej instancji bazy danych.
Przed każdym testem baza danych jest czyszczona, aby zapewnić niezależność przypadków testowych.
Testy są w pełni zautomatyzowane i można je uruchomić za pomocą pojedynczej komendy w środowisku Node.js, np. \texttt{npm test}, co zapewnia łatwość użycia i potencjalnej integracji z procesami CI/CD.

\section{Przypadki testowe dla zasobu Users}

\begin{itemize}
    \item \textbf{POST /api/users - Rejestracja użytkownika:}
    \begin{itemize}
        \item Sprawdzenie, czy nowy użytkownik może się zarejestrować z poprawnymi danymi (status 201).
        \item Sprawdzenie odrzucenia duplikatu nazwy użytkownika (status 409).
        \item Sprawdzenie odrzucenia duplikatu adresu e-mail (status 409).
        \item Sprawdzenie błędu dla nieprawidłowego formatu e-mail (status 400).
        \item Sprawdzenie odrzucenia zbyt krótkiej nazwy użytkownika (status 400).
        \item Sprawdzenie odrzucenia zbyt słabego hasła (status 400).
        \item Sprawdzenie odrzucenia żądania z brakującymi polami (status 400).
    \end{itemize}

    \item \textbf{GET /api/users - Lista użytkowników (tylko admin):}
    \begin{itemize}
        \item Sprawdzenie, czy administrator otrzymuje listę użytkowników (status 200, brak haseł w odpowiedzi).
        \item Sprawdzenie, że użytkownik bez uprawnień otrzymuje błąd 403.
        \item Sprawdzenie odrzucenia nieautoryzowanego dostępu (status 401).
    \end{itemize}

    \item \textbf{GET /api/users/:id - Pobieranie profilu użytkownika:}
    \begin{itemize}
        \item Sprawdzenie, czy użytkownik może zobaczyć własny profil (status 200, brak hasła).
        \item Sprawdzenie, że administrator może zobaczyć dowolny profil (status 200).
        \item Sprawdzenie odrzucenia dostępu do cudzego profilu przez zwykłego użytkownika (status 403).
        \item Sprawdzenie błędu 404 dla nieistniejącego użytkownika.
    \end{itemize}

    \item \textbf{PUT /api/users/:id - Aktualizacja użytkownika:}
    \begin{itemize}
        \item Sprawdzenie, czy użytkownik może zaktualizować własny profil (status 200).
        \item Sprawdzenie, że administrator może zaktualizować dowolny profil (status 200).
        \item Sprawdzenie odrzucenia próby aktualizacji cudzego profilu przez zwykłego użytkownika (status 403).
    \end{itemize}

    \item \textbf{DELETE /api/users/:id - Usuwanie użytkownika (tylko admin):}
    \begin{itemize}
        \item Sprawdzenie, czy administrator może usunąć użytkownika (status 204, użytkownik nie istnieje).
        \item Sprawdzenie, że zwykły użytkownik nie może usunąć profilu (status 403).
        \item Sprawdzenie błędu 404 dla nieistniejącego użytkownika.
    \end{itemize}
\end{itemize}

\section{Przypadki testowe dla zasobu Auth}

\begin{itemize}
    \item \textbf{POST /api/auth/login - Logowanie:}
    \begin{itemize}
        \item Sprawdzenie logowania z poprawnymi danymi (status 200).
        \item Sprawdzenie odrzucenia nieprawidłowego e-maila (status 401).
        \item Sprawdzenie odrzucenia nieprawidłowego hasła (status 401).
        \item Sprawdzenie odrzucenia brakujących danych logowania (status 401).
        \item Sprawdzenie odrzucenia pustego ciała żądania (status 401).
    \end{itemize}

    \item \textbf{Walidacja tokenu:}
    \begin{itemize}
        \item Sprawdzenie poprawności struktury tokenu JWT (status 200).
        \item Sprawdzenie odrzucenia nieprawidłowego tokenu (status 401).
        \item Sprawdzenie odrzucenia żądania bez tokenu (status 401).
    \end{itemize}
\end{itemize}

\section{Przypadki testowe dla zasobu Books}

\begin{itemize}
    \item \textbf{POST /api/books - Dodawanie książki (tylko admin):}
    \begin{itemize}
        \item Sprawdzenie, czy administrator może dodać książkę z poprawnymi danymi (status 201, dane zgodne z wysłanymi).
        \item Sprawdzenie, że użytkownik bez uprawnień administratora otrzymuje błąd 403 (Brak uprawnień administratora).
        \item Sprawdzenie odrzucenia żądania z brakującymi wymaganymi polami (tytuł, autor, ISBN) – status 400.
        \item Sprawdzenie, że próba dodania książki z istniejącym ISBN zwraca błąd 409 (Konflikt).
        \item Sprawdzenie odrzucenia nieautoryzowanego żądania – status 401.
    \end{itemize}

    \item \textbf{GET /api/books - Lista książek:}
    \begin{itemize}
        \item Sprawdzenie, czy lista książek jest zwracana bez autoryzacji (status 200, poprawna liczba książek i paginacja).
        \item Sprawdzenie wsparcia dla paginacji (strona 1, limit 2, poprawne dane paginacji).
        \item Sprawdzenie filtrowania książek po autorze (zwraca tylko książki danego autora).
        \item Sprawdzenie filtrowania książek po roku publikacji (zwraca książki z danego roku).
        \item Sprawdzenie sortowania książek po roku publikacji w porządku malejącym.
        \item Sprawdzenie obsługi pustych wyników dla nieistniejącego autora (status 200, pusta lista).
    \end{itemize}

    \item \textbf{GET /api/books/:id - Pobieranie pojedynczej książki:}
    \begin{itemize}
        \item Sprawdzenie, czy pojedyncza książka jest zwracana poprawnie (status 200, zgodne dane).
        \item Sprawdzenie błędu 404 dla nieistniejącej książki.
        \item Sprawdzenie obsługi nieprawidłowego ObjectId (status 500).
    \end{itemize}

    \item \textbf{PUT /api/books/:id - Aktualizacja książki (tylko admin):}
    \begin{itemize}
        \item Sprawdzenie, czy administrator może zaktualizować książkę (status 200, zgodne dane).
        \item Sprawdzenie, że użytkownik bez uprawnień administratora otrzymuje błąd 403.
        \item Sprawdzenie błędu 404 dla nieistniejącej książki.
    \end{itemize}

    \item \textbf{DELETE /api/books/:id - Usuwanie książki (tylko admin):}
    \begin{itemize}
        \item Sprawdzenie, czy administrator może usunąć książkę (status 204, książka nie istnieje w bazie).
        \item Sprawdzenie, że użytkownik bez uprawnień otrzymuje błąd 403.
        \item Sprawdzenie błędu 404 dla nieistniejącej książki.
    \end{itemize}
\end{itemize}

\section{Przypadki testowe dla zasobu Reviews}

\begin{itemize}
    \item \textbf{POST /api/reviews - Dodawanie recenzji:}
    \begin{itemize}
        \item Sprawdzenie, czy uwierzytelniony użytkownik może dodać recenzję (status 201, zgodne dane).
        \item Sprawdzenie odrzucenia nieautoryzowanej recenzji (status 401).
        \item Sprawdzenie odrzucenia recenzji z brakującymi polami (status 400).
        \item Sprawdzenie odrzucenia nieprawidłowych wartości oceny (poza zakresem 1–5, status 400).
        \item Sprawdzenie błędu 404 dla nieistniejącej książki.
        \item Sprawdzenie odrzucenia duplikatu recenzji tego samego użytkownika dla tej samej książki (status 409).
        \item Sprawdzenie, że różni użytkownicy mogą oceniać tę samą książkę (status 201).
    \end{itemize}

    \item \textbf{GET /api/reviews - Lista recenzji:}
    \begin{itemize}
        \item Sprawdzenie, czy wszystkie recenzje są zwracane bez autoryzacji (status 200, posortowane po dacie).
        \item Sprawdzenie filtrowania recenzji po \texttt{book\_id} (tylko recenzje danej książki).
        \item Sprawdzenie poprawności dołączonych danych użytkownika i książki w odpowiedzi.
        \item Sprawdzenie obsługi pustych wyników dla nieistniejącej książki (status 200, pusta lista).
    \end{itemize}

    \item \textbf{GET /api/reviews/:id - Pobieranie pojedynczej recenzji:}
    \begin{itemize}
        \item Sprawdzenie, czy pojedyncza recenzja jest zwracana z poprawnymi danymi oraz uzupełnionymi informacjami o użytkowniku i książce (status 200).
        \item Sprawdzenie błędu 404 dla nieistniejącej recenzji.
    \end{itemize}

    \item \textbf{PUT /api/reviews/:id - Aktualizacja recenzji:}
    \begin{itemize}
        \item Sprawdzenie, czy autor może zaktualizować własną recenzję (status 200).
        \item Sprawdzenie, że administrator może zaktualizować dowolną recenzję (status 200).
        \item Sprawdzenie odrzucenia aktualizacji przez innego użytkownika (status 403).
        \item Sprawdzenie odrzucenia nieprawidłowej oceny (status 400).
        \item Sprawdzenie błędu 404 dla nieistniejącej recenzji.
        \item Sprawdzenie odrzucenia nieautoryzowanej aktualizacji (status 401).
    \end{itemize}

    \item \textbf{DELETE /api/reviews/:id - Usuwanie recenzji:}
    \begin{itemize}
        \item Sprawdzenie, czy autor może usunąć własną recenzję (status 204, recenzja nie istnieje).
        \item Sprawdzenie, że administrator może usunąć dowolną recenzję (status 204).
        \item Sprawdzenie odrzucenia usuwania przez innego użytkownika (status 403).
        \item Sprawdzenie błędu 404 dla nieistniejącej recenzji.
        \item Sprawdzenie odrzucenia nieautoryzowanego usuwania (status 401).
    \end{itemize}

    \item \textbf{Przypadki brzegowe i obsługa błędów:}
    \begin{itemize}
        \item Sprawdzenie obsługi nieprawidłowych ObjectId (status 500).
        \item Sprawdzenie ścisłego sprawdzania granic oceny (0.9 i 5.1 odrzucone, 1–5 akceptowane).
    \end{itemize}
\end{itemize}

\end{document}