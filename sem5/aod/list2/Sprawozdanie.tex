\documentclass{article}
\usepackage{polski}
\usepackage{float}
% \usepackage{adjustbox}
% \usepackage{tikz}
% \usepackage{graphicx}
\usepackage{amsmath}
\usepackage{amssymb}
\usepackage{array}
\usepackage{caption}
\usepackage{xcolor}

\title{Sprawozdanie 2 - Algorytmy Optymalizacji Dyskretnej}
\author{Michał Kallas}

\begin{document}

\maketitle

\section{Zadanie 1}
\subsection{Opis problemu}

Rozważmy problem zakupu paliwa od dostawców i jego dystrybucji na lotniska, aby zminimalizować łączny koszt dostaw przy uwzględnieniu ograniczeń dostępności oraz zapotrzebowania na paliwo.

\subsection{Opis modelu}
\subsubsection{Parametry}
\begin{itemize}
    \item $n$ - liczba dostawców paliwa
    \item $m$ - liczba lotnisk
    \item $f_i$ - maksymalna ilość paliwa, jaką może dostarczyć dostawca $i$
    \item $d_j$ - wymagana ilość paliwa na lotnisku $j$
    \item $c_{ij}$ - koszt dostarczenia jednego galonu paliwa przez dostawcę $i$ na lotnisko $j$ (w \$).
\end{itemize}

\subsubsection{Zmienne decyzyjne}
\begin{itemize}
    \item $x_{ij} \geq 0$ - ilość paliwa (w galonach) dostarczonego przez dostawcę $i$ na lotnisko $j$.
\end{itemize}

\subsubsection{Ograniczenia}
\begin{itemize}
    \item Każdy dostawca $i$ nie może dostarczyć więcej paliwa, niż posiada:
    \[
    \sum_{j=1}^{m} x_{ij} \leq f_i
    \]
    \item Każde lotnisko $j$ musi otrzymać dokładnie tyle paliwa, ile wynosi jego zapotrzebowanie:
    \[
    \sum_{i=1}^{n} x_{ij} = d_j
    \]
\end{itemize}

\subsubsection{Funkcja celu}
Celem jest zminimalizowanie całkowitego kosztu dostaw paliwa. Funkcja celu ma postać:
\[
\min \sum_{i=1}^{n} \sum_{j=1}^{m} c_{ij} x_{ij}
\]

\subsection{Egzemplarz z zadania}
\begin{itemize}
    \item $n = 3$ (trzech dostawców),
    \item $m = 4$ (cztery lotniska),
    \item Dostępne ilości paliwa od dostawców:
    \[
    f = [275000, 550000, 660000]
    \]
    \item Wymagane ilości paliwa na lotniskach:
    \[
    d = [110000, 220000, 330000, 440000]
    \]
    \item Koszty dostarczenia jednego galonu paliwa ($c_{ij}$) przez dostawców na lotniska:
\end{itemize}

\begin{table}[h!]
    \centering
    \begin{tabular}{|c|c|c|c|}
        \hline
        & Firma 1 & Firma 2 & Firma 3\\
        \hline
        Lotnisko 1 & 10 & 7 & 8\\
        \hline
        Lotnisko 2 & 10 & 11 & 14\\
        \hline
        Lotnisko 3 & 9 & 12 & 4\\
        \hline
        Lotnisko 4 & 11 & 13 & 9\\
        \hline
    \end{tabular}
\end{table}


\subsection{Rozwiązanie}
Suma dostaw paliwa przez poszczególne firmy:
\begin{itemize}
    \item Firma 1: $275\,000$ galonów,
    \item Firma 2: $165\,000$ galonów,
    \item Firma 3: $660\,000$ galonów.
\end{itemize}

\begin{table}[H]
    \centering
    \caption{Ilość paliwa dostarczanego przez firmy na lotniska (w galonach)}
    \begin{tabular}{|c|c|c|c|c|}
        \hline
        & Lotnisko 1 & Lotnisko 2 & Lotnisko 3 & Lotnisko 4 \\ \hline
        Firma 1 & 0 & 165 000 & 0 & 110 000 \\ \hline
        Firma 2 & 110 000 & 55 000 & 0 & 0 \\ \hline
        Firma 3 & 0 & 0 & 330 000 & 330 000 \\ \hline
    \end{tabular}
\end{table}

\begin{enumerate}
    \item[(a)] Jaki jest minimalny łączny koszt dostaw wymaganych ilości paliwa na wszystkie lotniska? - \textbf{\$8\,525\,000}.

    \item[(b)] Czy wszystkie firmy dostarczają paliwo? - \textbf{Tak}.

    \item[(c)] Czy możliwości dostaw paliwa przez firmy są wyczerpane? - \textbf{Są wyczerpane dla firm 1 i 3}.
\end{enumerate}

\section{Zadanie 2}
\subsection{Opis problemu}
Zakład produkcyjny może produkować $n$ różne wyroby \( P_i \). Każdy z wyrobów wymaga pewnego czasu obróbki na $m$ maszynach. Każda z maszyn jest dostępna przez pewną ilość godzin w tygodniu. Produkty mają określoną cenę sprzedaży oraz koszty zmienne związane z ich produkcją. Należy wyznaczyć optymalny tygodniowy plan produkcji, maksymalizując zysk.

\subsection{Opis modelu}
\subsubsection{Parametry}
\begin{itemize}
    \item \( n \) - liczba produktów,
    \item \( m \) - liczba maszyn,
    \item \( p_i \) - cena sprzedaży kilogramu produktu \( P_i \),
    \item \( c_i \) - koszt materiałowy na kilogram produktu \( P_i \),
    \item \( d_i \) - maksymalny tygodniowy popyt na produkt \( P_i \) (w kilogramach),
    \item \( t_{ij} \) - czas obróbki produktu \( P_i \) na maszynie \( M_j \) (w minutach na kilogram wyrobu),
    \item \( a_j \) - dostępny czas pracy maszyny \( M_j \) w godzinach na tydzień,
    \item \( k_j \) - koszt pracy maszyny \( M_j \) na godzinę.
\end{itemize}

\subsubsection{Zmienne decyzyjne}
\begin{itemize}
    \item \( x_i \geq 0 \) - ilość wyprodukowanego produktu \( P_i \) (w kilogramach).
\end{itemize}

\subsubsection{Ograniczenia}
\begin{itemize}
    \item Ilość wyprodukowanych produktów \( P_i \) nie może przekroczyć maksymalnego popytu na ten produkt:
    \[
    x_i \leq d_i
    \]
    \item Czas pracy maszyn nie może przekroczyć dostępnego tygodniowego czasu:
    \[
    \sum_{i=1}^{m} t_{ij} x_i \leq a_j \cdot 60
    \]
\end{itemize}

\subsubsection{Funkcja celu}
Celem jest maksymalizacja zysku, który jest różnicą między przychodem ze sprzedaży produktów a kosztami materiałowymi i kosztami pracy maszyn. Funkcja celu ma postać:
\[
\max \sum_{i=1}^{n} (p_i - c_i) x_i - \sum_{j=1}^{m} \frac{k_j}{60} \left( \sum_{i=1}^{n} t_{ij} \cdot x_i \right)
\]

\subsection{Egzemplarz z zadania}
\begin{itemize}
    \item Cena sprzedaży produktów za kilogram: \( p = [9, 7, 6, 5] \),
    \item Koszty materiałowe za kilogram: \( c = [4, 1, 1, 1] \),
    \item Maksymalny tygodniowy popyt w kilogramach: \( d = [400, 100, 150, 500] \),
    \item Czas obróbki na maszynach (w minutach na kilogram produktu):
        \begin{table}[H]
            \centering
            \begin{tabular}{|c|c|c|c|}
                \hline
                & Maszyna M1 & Maszyna M2 & Maszyna M3 \\
                \hline
                Produkt P1 & 5 & 10 & 6 \\
                \hline
                Produkt P2 & 3 & 6 & 4 \\
                \hline
                Produkt P3 & 4 & 5 & 3 \\
                \hline
                Produkt P4 & 4 & 2 & 1 \\
                \hline
            \end{tabular}
        \end{table}
    \item Czas pracy dostępny dla każdej maszyny w godzinach na tydzień: \( a = [60, 60, 60] \),
    \item Koszty pracy maszyn za godzinę: \( k = [2, 2, 3] \).
\end{itemize}

\subsection{Rozwiązanie}
\begin{itemize}
    \item Optymalna liczba wyprodukowanych produktów:
    \[
    x = [125, 100, 150, 500]
    \]
    Oznacza to, że należy wyprodukować:
    \begin{itemize}
        \item \textbf{125kg} produktu \( P_1 \),
        \item \textbf{100kg} produktu \( P_2 \),
        \item \textbf{150kg} produktu \( P_3 \),
        \item \textbf{500kg} produktu \( P_4 \).
    \end{itemize}
    Widzimy, że dla pierwszego produktu nie został osiągnięty cały popyt.

    \item Maksymalny zysk wynosi:
    \textbf{\$3632.5}
\end{itemize}

\section{Zadanie 3}
\subsection{Opis problemu}
W zadaniu przedstawiono problem produkcji i magazynowania towaru w różnych okresach. Celem jest minimalizacja łącznego kosztu produkcji oraz magazynowania towaru przy spełnieniu zapotrzebowania na towar w każdym z okresów.

\subsection{Opis modelu}
\subsubsection{Parametry}
\begin{itemize}
    \item $K$ - liczba okresów,
    \item $c_j$ - koszt produkcji jednej jednostki towaru w okresie $j$,
    \item $o_j$ - koszt produkcji jednej ponadwymiarowej jednostki w okresie $j$,
    \item $d_j$ - zapotrzebowanie na towar w okresie $j$,
    \item $n_j$ - maksymalna liczba jednostek produkowanych w trybie normalnej produkcji w okresie $j$,
    \item $a_j$ - maksymalna liczba jednostek produkowanych w trybie ponadwymiarowym w okresie $j$,
    \item $m_0$ - początkowa liczba jednostek w magazynie,
    \item $m_{\text{max}}$ - maksymalna liczba jednostek, które mogą być przechowywane w magazynie,
    \item $s$ - koszt przechowywania jednej jednostki w magazynie przez jeden okres.
\end{itemize}

\subsubsection{Zmienne decyzyjne}
\begin{itemize}
    \item $x_j \geq 0$ - liczba jednostek wyprodukowanych w trybie normalnej produkcji w okresie $j$,
    \item $y_j \geq 0$ - liczba jednostek wyprodukowanych w trybie ponadwymiarowej produkcji w okresie $j$,
    \item $m_j \geq 0$ - liczba jednostek przechowywanych w magazynie na koniec okresu $j - 1$.
\end{itemize}

\subsubsection{Ograniczenia}
\begin{itemize}
    \item Liczba jednostek wyprodukowanych w trybie normalnym nie może przekroczyć dozwolonej maksymalnej liczby jednostek w okresie $j$:
    \[
    x_j \leq n_j
    \]
    \item Liczba jednostek wyprodukowanych w trybie ponadwymiarowym nie może przekroczyć dozwolonej maksymalnej liczby jednostek w okresie $j$:
    \[
    y_j \leq a_j
    \]
    \item Liczba jednostek w magazynie nie może przekroczyć maksymalnej pojemności magazynu:
    \[
    m_j \leq m_{\text{max}}
    \]
    \item Liczba wyprodukowanych jednostek i jednostek przechowywanych w magazynie musi zaspokoić zapotrzebowanie w każdym okresie:
    \[
    x_j + y_j + m_j - m_{j + 1} = d_j
    \]
    \item Magazyn na początku pierwszego okresu przechowuje początkową ilość jednostek:
    \[
    m_1 = m_0
    \]
\end{itemize}

\subsubsection{Funkcja celu}
Celem jest minimalizacja całkowitego kosztu produkcji i magazynowania, który można zapisać jako:
\[
\min \sum_{j=1}^{K} (c_j x_j + o_j y_j + s m_j)
\]
gdzie:
\begin{itemize}
    \item $c_j \cdot x_j$ - koszt produkcji jednostek w trybie normalnym,
    \item $o_j \cdot y_j$ - koszt produkcji jednostek ponadwymiarowych,
    \item $s \cdot m_j$ - koszt przechowywania jednostek w magazynie.
\end{itemize}

\subsection{Egzemplarz z zadania}
\begin{itemize}
    \item Liczba okresów: \( K = 4 \),
    \item Początkowa liczba produktów w magazynie: \( m_0 = 15 \),
    \item Maksymalna liczba produktów w magazynie: \( m_{max} = 70 \),
    \item Koszty, popyt i maksymalna produkcja w kolejnych okresach:
    \[
    \begin{array}{|c|c|c|c|c|c|}
    \hline
    j & c_j (\$) & o_j (\$) & d_j (\text{jednostki}) & n_j (\text{jednostki}) & a_j (\text{jednostki})\\
    \hline
    1 & 6000 & 8000 & 130 & 100 & 60\\
    2 & 4000 & 6000 & 80 & 100 & 65\\
    3 & 8000 & 10000 & 125 & 100 & 70\\
    4 & 9000 & 11000 & 195 & 100 & 60\\
    \hline
    \end{array}
    \]
\end{itemize}


\subsection{Rozwiązanie}
Po rozwiązaniu problemu, uzyskujemy następujące wyniki:

\begin{table}[h!]
    \centering
    \begin{tabular}{|c|c|c|c|}
        \hline
        $j$ & $x_j$ & $y_j$ & $m_j$ \\
        \hline
        1 & 100 & 15 & 15 \\
        2 & 100 & 50 & 0 \\
        3 & 100 & 0 & 70 \\
        4 & 100 & 50 & 45 \\
        \hline
    \end{tabular}
    \caption{Wyniki dla zmiennych decyzyjnych w kolejnych okresach}
\end{table}


\begin{itemize}
    \item[(a)] Jaki jest minimalny łączny koszt produkcji i magazynowania towaru? - \textbf{\$3\,842\,500}.

    \item[(b)] W których okresach firma musi zaplanować produkcję ponadwymiarową? - \textbf{w okresach 1, 2 i 4}.

    \item[(c)] W których okresach możliwości magazynowania towaru są wyczerpane? - \textbf{na koniec 2 okresu}.
\end{itemize}

\section{Zadanie 4}
\subsection{Opis problemu}
Dla sieci miast reprezentowanej przez skierowany graf $G = (N, A)$, w której $N$ oznacza zbiór miast, a $A$ zbiór połączeń między miastami, należy znaleźć najtańszą ścieżkę między wybranymi miastami $i^\circ$ i $j^\circ$. Każde połączenie $(i, j) \in A$ ma przypisany koszt przejazdu $c_{ij}$ oraz czas przejazdu $t_{ij}$. Celem jest zminimalizowanie kosztu podróży z miasta $i^\circ$ do $j^\circ$, przy czym całkowity czas podróży nie może przekroczyć określonej wartości $T$.

\subsection{Opis modelu}
\subsubsection{Parametry}
\begin{itemize}
    \item $N$ - zbiór miast,
    \item $A$ - zbiór połączeń (łuków) między miastami,
    \item $c_{ij}$ - koszt przejazdu dla połączenia $(i, j) \in A$,
    \item $t_{ij}$ - czas przejazdu dla połączenia $(i, j) \in A$,
    \item $T$ - maksymalny dopuszczalny czas przejazdu,
    \item $i^\circ$, $j^\circ$ - miasto początkowe i końcowe.
\end{itemize}

\subsubsection{Zmienne decyzyjne}
\begin{itemize}
    \item $x_{ij} \in \{0, 1\}$ - zmienna decyzyjna, przyjmująca wartość 1, jeśli połączenie $(i, j) \in A$ jest wykorzystywane w ścieżce, oraz 0 w przeciwnym razie.
\end{itemize}

\subsubsection{Ograniczenia}
\begin{itemize}
    \item Całkowity czas przejazdu nie może przekraczać maksymalnego czasu $T$:
    \[
    \sum_{(i, j) \in A} t_{ij} \cdot x_{ij} \leq T
    \]
    \item Nie można korzystać z nieistniejących połączeń:
    \[
    x_{ij} = 0, \quad \forall (i, j) \notin A
    \]
    \item Miasto początkowe musi mieć dokładnie jedno połączenie wychodzące:
    \[
    \sum_{j \in N} x_{i^\circ j} = 1
    \]
    \item Miasto końcowe musi mieć dokładnie jedno połączenie wchodzące:
    \[
    \sum_{i \in N} x_{i j^\circ} = 1
    \]
    \item Dla każdego miasta, z wyjątkiem początkowego i końcowego, liczba połączeń wchodzących musi być równa liczbie połączeń wychodzących:
    \[
    \sum_{j \in N} x_{ij} = \sum_{j \in N} x_{ji}, \quad \forall i \in N \setminus \{i^\circ, j^\circ\}
    \]
\end{itemize}

\subsubsection{Funkcja celu}
Minimalizacja całkowitego kosztu przejazdu:
\[
\min \sum_{(i, j) \in A} c_{ij} \cdot x_{ij}
\]

\subsection{Egzemplarze}
\subsubsection{Egzemplarz z zadania}
Egzemplarz przedstawiony w poleceniu dla $N = \{1, \ldots, 10\}$, $i^\circ = 1$, $j^\circ = 10$, $T = 15$, z krawędziami:

\begin{table}[H]
\centering
\begin{tabular}{|c|c|c|c|}
\hline
$(i, j)$ & $c_{ij}$ & $t_{ij}$ \\
\hline
(1, 2) & 3 & 4 \\
(1, 3) & 4 & 9 \\
(1, 4) & 7 & 10 \\
(1, 5) & 8 & 12 \\
(2, 3) & 2 & 3 \\
(3, 4) & 4 & 6 \\
(3, 5) & 2 & 2 \\
(3, 10) & 6 & 11 \\
(4, 5) & 1 & 1 \\
(4, 7) & 3 & 5 \\
(5, 6) & 5 & 6 \\
(5, 7) & 3 & 3 \\
(5, 10) & 5 & 8 \\
(6, 1) & 5 & 8 \\
(6, 7) & 2 & 2 \\
(6, 10) & 7 & 11 \\
(7, 3) & 4 & 6 \\
(7, 8) & 3 & 5 \\
(7, 9) & 1 & 1 \\
(8, 9) & 1 & 2 \\
(9, 10) & 2 & 2 \\
\hline
\end{tabular}
\caption{Koszt i czas przejazdu między miastami dla egzemplarza z zadania}
\end{table}

\subsubsection{Mój egzemplarz}
Egzemplarz wymyślony przez mnie dla $N = \{1, \ldots, 10\}$, $i^\circ = 1$, $j^\circ = 10$, $T = 15$, z krawędziami:

\begin{table}[H]
\centering
\begin{tabular}{|c|c|c|c|}
\hline
$(i, j)$ & $c_{ij}$ & $t_{ij}$ \\
\hline
(1, 2) & 1 & 3 \\
(2, 3) & 2 & 2 \\
(3, 10) & 3 & 5 \\
(1, 4) & 1 & 10 \\
(4, 10) & 1 & 6 \\
\hline
\end{tabular}
\caption{Koszt i czas przejazdu między miastami dla mojego egzemplarza}
\label{tab:edges}
\end{table}

\subsection{Rozwiązanie}
\subsection{Rozwiązanie egzemplarza z zadania (a)}
\begin{itemize}
    \item Całkowity koszt przejazdu: 13
    \item Całkowity czas przejazdu: 15
    \item Wykorzystane połączenia: (1,2), (2,3), (3,5), (5,7), (7,9), (9,10)
\end{itemize}

\subsection{Rozwiązanie mojego egzemplarza (b)}
\begin{itemize}
    \item Całkowity koszt przejazdu: 6
    \item Całkowity czas przejazdu: 10
    \item Wykorzystane połączenia: (1,2), (2,3), (3,10)
\end{itemize}

\subsection{Odpowiedzi na pytania}

\begin{enumerate}
    \item[(a)] Czy ograniczenie na całkowitoliczbowość zmiennych decyzyjnych jest potrzebne? Jeżeli nie, to uzasadnij dlaczego. Jeżeli tak, to zaproponuj kontrprzykład, w którym po usunięciu ograniczenia na całkowitoliczbowość (tj. mamy przypadek, w którym model jest modelem programowania liniowego) zmienne decyzyjne w rozwiązaniu optymalnym nie mają wartości całkowitych.

    \textbf{Tak, ograniczenie na całkowitoliczbowość zmiennych decyzyjnych jest konieczne, ponieważ zmienne \(x[i,j]\) reprezentują obecność krawędzi w ścieżce i muszą przyjmować wartości 0 lub 1. Usunięcie tego ograniczenia mogłoby prowadzić do wartości ułamkowych zmiennych, co nie ma sensu w kontekście problemu. Chociażby dla mojego grafu, po zdjęciu tego ograniczenia $x[i, j]$ zaczyna przyjmować wartości ułamkowe.}

    \item[(b)] Czy po usunięciu ograniczenia na czasy przejazdu w modelu bez ograniczeń na całkowitoliczbowość zmiennych decyzyjnych i rozwiązaniu problemu otrzymane połączenie zawsze jest akceptowalnym rozwiązaniem? Uzasadnij odpowiedź.

    \textbf{W tym wypadku ograniczenie na całkowitoliczbowość nie jest potrzebne.}
\end{enumerate}

\section{Zadanie 5}
\subsection{Opis problemu}
Mamy pewną ilość dzielnic, do których potrzebujemy przypisać radiowozy na konkretne zmiany.
Musimy wziąć pod uwagę minimalne wymagania dla dzielnic i zmian, a także minimalną i maksymalną liczbę radiowozów dla danej dzielnicy i zmiany.

\subsection{Opis modelu}

\subsubsection{Parametry}
\begin{itemize}
    \item $n$ - ilość dzielnic,
    \item $m$ - ilość zmian,
    \item $rMIN_{ij}$ - minimalna liczba radiowozów dla dzielnicy $p_i$ na zmianie $s_j$,
    \item $rMAX_{ij}$ - maksymalna liczba radiowozów dla dzielnicy $p_i$ na zmianie $s_j$,
    \item $dMIN_i$ - minimalna liczba radiowozów dostępnych dla dzielnicy $p_i$,
    \item $zMIN_j$ - minimalna liczba radiowozów dostępnych na zmianie $s_j$.
\end{itemize}

\subsubsection{Zmienne decyzyjne}
\begin{itemize}
    \item $x_{ij}$ - liczba radiowozów przydzielonych do dzielnicy $p_i$ w zmianie $s_j$.
\end{itemize}

\subsubsection{Ograniczenia}
\begin{itemize}
    \item Minimalna i maksymalna liczba radiowozów dla każdej dzielnicy i zmiany:
    \[
    rMIN_{ij} \leq x_{ij} \leq rMAX_{ij}
    \]
    \item Minimalna liczba radiowozów dla każdej dzielnicy:
    \[
    \sum_{j=1}^{m} x_{ij} \geq dMIN_i
    \]
    \item Minimalna liczba radiowozów w każdej zmianie:
    \[
    \sum_{i=1}^{n} x_{ij} \geq zMIN_j
    \]
\end{itemize}

\subsubsection{Funkcja celu}
Minimalizujemy całkowitą liczbę radiowozów:
\[
\min \sum_{i=1}^{n} \sum_{j=1}^{m} x_{ij}.
\]

\subsection{Egzemplarz}
\begin{itemize}
    \item Liczba dzielnic: \( n = 3 \),
    \item Liczba zmian: \( m = 3 \),
    \item Minimalne i maksymalne liczby radiowozów dla każdej zmiany i dzielnicy:
        \begin{table}[H]
            \centering
            \begin{tabular}{|c|c|c|c|}
                \hline
                & Zmiana 1 & Zmiana 2 & Zmiana 3 \\
                \hline
                Dzielnica $p_1$ (min) & 2 & 4 & 3 \\
                Dzielnica $p_2$ (min) & 3 & 6 & 5 \\
                Dzielnica $p_3$ (min) & 5 & 7 & 6 \\
                \hline
                Dzielnica $p_1$ (max) & 3 & 7 & 5 \\
                Dzielnica $p_2$ (max) & 5 & 7 & 10 \\
                Dzielnica $p_3$ (max) & 8 & 12 & 10 \\
                \hline
            \end{tabular}
        \end{table}

    \item Minimalna liczba radiowozów na zmianę: 10, 20, 18 dla zmiany 1, 2 i 3.

    \item Minimalna liczba radiowozów na dzielnicę: 10, 14, 13 dla dzielnic $p_1$, $p_2$, $p_3$.
\end{itemize}

\subsection{Rozwiązanie}
Po rozwiązaniu problemu, uzyskujemy następujące liczby radiowozów:

\begin{table}[H]
    \centering
    \begin{tabular}{|c|c|c|c|}
        \hline
        & Zmiana 1 & Zmiana 2 & Zmiana 3 \\
        \hline
        Dzielnica $p_1$ & 2 & 7 & 5 \\
        Dzielnica $p_2$ & 3 & 6 & 7 \\
        Dzielnica $p_3$ & 5 & 7 & 6 \\
        \hline
    \end{tabular}
\end{table}

\noindent \textbf{Minimalna łączna liczba radiowozów: 48.}

\section{Zadanie 6}
\subsection{Opis zadania}
Firma przeładunkowa składuje kontenery na terenie podzielonym na siatkę kwadratów, z których każdy może być zajęty przez co najwyżej jeden kontener.
W celu monitorowania kontenerów firma musi rozmieszczać kamery, które mogą obserwować kwadraty w określonym zasięgu poziomym i pionowym, przy czym kamera nie może stać na kwadracie zajętym przez kontener.
Celem jest rozmieszczenie minimalnej liczby kamer tak, aby każdy kontener był monitorowany przez co najmniej jedną kamerę.

\subsection{Opis modelu}
\subsubsection{Parametry}
\begin{itemize}
    \item $m$ - liczba wierszy terenu składowiska,
    \item $n$ - liczba kolumn terenu składowiska,
    \item $k$ - zasięg obserwacji kamery (w liczbie kwadratów) w każdym kierunku,
    \item $C_{ij}$ - macierz (o wymiarach $m \times n$) reprezentująca pozycje kontenerów. $C_{ij} = 1$, jeśli na kwadracie $(i,j)$ znajduje się kontener, a $C_{ij} = 0$ w przeciwnym wypadku.
\end{itemize}

\subsubsection{Zmienne decyzyjne}
\begin{itemize}
    \item $x_{ij}$ - zmienna, która przyjmuje wartość 1, jeśli w kwadracie $(i,j)$ umieszczona jest kamera, i 0 w przeciwnym wypadku.
\end{itemize}

\subsubsection{Ograniczenia}
\begin{itemize}
    \item Kamery mogą być umieszczane jedynie na pustych kwadratach:
    \[
    x_{ij} = 0, \quad \text{dla wszystkich } (i, j) \text{, dla których } C_{ij} = 1.
    \]
    \item Każdy kwadrat z kontenerem musi być monitorowany przez co najmniej jedną kamerę w jego zasięgu:
    \[
    \sum_{a = \max(i - k, 1)}^{\min(i + k, m)} x_{aj} + \sum_{b = \max(j - k, 1)}^{\min(j + k, n)} x_{ib} \geq 1, \quad \text{dla wszystkich } (i,j) \text{, dla których } C_{ij} = 1.
    \]
\end{itemize}

\subsubsection{Funkcja celu}
Minimalizujemy liczbę kamer umieszczonych na terenie składowiska:
\[
\min \sum_{i=1}^{m} \sum_{j=1}^{n} x_{ij}.
\]

\subsection{Egzemplarz}
Jako egzemplarz zadania przyjęto parametry $m = 5$ i $n = 5$ z następującą macierzą $C_{ij}$:

\begin{table}[H]
    \centering
    \begin{tabular}{|c|c|c|c|c|}
        \hline
        \textcolor{blue}{1} & 0 & \textcolor{blue}{1} & 0 & \textcolor{blue}{1} \\
        0 & 0 & 0 & 0 & 0 \\
        0 & \textcolor{blue}{1} & 0 & \textcolor{blue}{1} & 0 \\
        0 & 0 & \textcolor{blue}{1} & 0 & 0 \\
        \textcolor{blue}{1} & 0 & 0 & 0 & \textcolor{blue}{1} \\
        \hline
    \end{tabular}
    \caption{Macierz kontenerów $C_{ij}$, kontenery zaznaczone na niebiesko}
\end{table}

\subsection{Rozwiązanie}
Poniżej przedstawiam rozmieszczenie kamer $x_{ij}$ wraz z macierzą kontenerów $C_{ij}$ dla kolejnych wartości $k$. Kontenery zostały zaznaczone na niebiesko, a kamery na czerwono:\\

\centering \textbf{k = 1, Minimalna liczba kamer: 5}
\begin{table}[H]
    \centering
    \begin{tabular}{|c|c|c|c|c|}
        \hline
        \textcolor{blue}{1} & \textcolor{red}{1} & \textcolor{blue}{1} & \textcolor{red}{1} & \textcolor{blue}{1} \\
        0 & 0 & 0 & 0 & 0 \\
        0 & \textcolor{blue}{1} & \textcolor{red}{1} & \textcolor{blue}{1} & 0 \\
        \textcolor{red}{1} & 0 & \textcolor{blue}{1} & 0 & 0 \\
        \textcolor{blue}{1} & 0 & 0 & \textcolor{red}{1} & \textcolor{blue}{1} \\
        \hline
    \end{tabular}
\end{table}

\centering \textbf{k = 2, Minimalna liczba kamer: 3}
\begin{table}[H]
    \centering
    \begin{tabular}{|c|c|c|c|c|}
        \hline
        \textcolor{blue}{1} & 0 & \textcolor{blue}{1} & 0 & \textcolor{blue}{1} \\
        0 & 0 & 0 & 0 & 0 \\
        \textcolor{red}{1} & \textcolor{blue}{1} & \textcolor{red}{1} & \textcolor{blue}{1} & \textcolor{red}{1} \\
        0 & 0 & \textcolor{blue}{1} & 0 & 0 \\
        \textcolor{blue}{1} & 0 & 0 & 0 & \textcolor{blue}{1} \\
        \hline
    \end{tabular}
\end{table}

\end{document}
