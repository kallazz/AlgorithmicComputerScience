\documentclass{article}
\usepackage{polski}
\usepackage{mathtools}
\usepackage{float}

\title{Zadanie 2. z 6. listy zadań z JFTT}
\author{Michał Kallas}

\setlength\parindent{0pt}
\begin{document}

\maketitle

\section{Treść zadania}
Wyeliminuj lewostronną rekursję z gramatyki:
$$
\begin{aligned}
S &\to (L) \mid a \\
L &\to L, S \mid S
\end{aligned}
$$

\section{Lewostronna rekurencja}
Gramatyka jest \emph{lewostronnie rekurencyjna}, jeśli istnieje nieterminal $A$ taki, że istnieje wyprowadzenie $A \Rightarrow^+ A\alpha$ dla pewnego $\alpha$.

\section{Algorytm}
Aby wyeliminować lewostronną rekurencję, można skorzystać z prostego algorytmu.
Dla produkcji w postaci
${\displaystyle A \rightarrow A\alpha_1 \mid \ldots \mid A\alpha_n \mid \beta_1 \mid \ldots \mid \beta_m}$, gdzie:

\begin{itemize}
    \item $\alpha$ jest niepustym ciągiem nieterminali i terminali
    \item $\beta$ jest ciągiem nieterminali i terminali, który nie zaczyna się od $A$
\end{itemize}

Dodajemy nieterminal $A'$ i przekształcamy ją następująco:
$$
A \rightarrow \beta_1A' \mid \ldots \mid \beta_mA'
$$
$$
A' \rightarrow \alpha_1A' \mid \ldots \mid \alpha_nA' \mid \epsilon
$$
Powtarzamy ten proces, aż nie pozostanie żadna bezpośrednia lewostronna rekurencja.

\section{Rozwiązanie}
W zadanej gramatyce lewostronna rekurencja występuje w produkcji $L$:
$$
\begin{aligned}
S &\to (L) \mid a \\
L &\to \boldsymbol{L}, S \mid S
\end{aligned}
$$
Zacznijmy od przekształcenia produkcji $L$ na formę z nowym nieterminalem $L'$:
$$
L \rightarrow SL'
$$
Teraz zdefiniujmy $L'$:
$$
L' \rightarrow ,SA' \mid \epsilon
$$

W ten sposób usunęliśmy lewostronną rekurencję z zadanej gramatyki.
Nowa gramatyka prezentuje się następująco:
$$
\begin{aligned}
S &\to (L) \mid a \\
L &\to SL' \\
L' &\to ,SL' \mid \epsilon
\end{aligned}
$$

\end{document}
